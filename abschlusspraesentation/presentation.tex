\documentclass[t]{beamer}
\usetheme[deutsch]{KIT}
\setbeamercovered{transparent}
\setbeamertemplate{navigation symbols}{}

\KITfoot{YUV.KA - Praxis der Softwareentwicklung WS 11/12}
\usepackage[utf8]{inputenc}
\usepackage{ngerman}
\usepackage{graphicx}
\usenavigationsymbols

\title{YUV.KA}
\subtitle{Abschlusspräsentation}
\author{Max Wagner $\cdot$ Patrick Gemander $\cdot$ Sebastian Ullrich $\cdot$ Michael Vollmer \\ Robert Hangu $\cdot$ Daniel Lebert}

\institute[ITEC]{Institut für Technische Informatik}

\TitleImage[trim = -20cm 0 0 0,height=\titleimageht]{logo.png}

\begin{document}

\begin{frame}
	\maketitle
\end{frame}

\begin{frame}
	\frametitle{Was ist ein Videoencoder?}
	
	\begin{center}
		\vspace*{\fill}
		\includegraphics[scale=.43]{video-encoding-process.png}
		\vspace*{\fill}
		\onslide<2-> Warum das Ganze? ~\\
		\onslide<3-> $ \Longrightarrow $ Speicherbedarf der Videos reduzieren
	\end{center}
\end{frame}

\begin{frame}
	\frametitle{Funktionsweise des Encoders}
	\begin{minipage}{5.5cm}
		\includegraphics[scale=.29]{overlay-medium.png}	
	\end{minipage}
	\begin{minipage}{5.5cm}
		\begin{itemize}
			\item<+-> Unterteilung in Makroblöcke
			\item<+-> Wiederverwendung von Bildteilen aus dem aktuellen Frame
			\item<+-> Wiederverwendung von Bildteilen aus vorherigen Frames
		\end{itemize}
	\end{minipage}
\end{frame}

\begin{frame}
	\frametitle{Der Referenzencoder}
	~\\
	% Referenz-Encoder
	\begin{minipage}{5.3cm}
		\includegraphics[scale=.5]{ref.png}
		~\\
		\begin{itemize}
			\item<2-> Liefert optimale Lösung
			\item<4-> Geht alle Möglichkeiten durch
			\item<5-> Sehr langsam
		\end{itemize}
	\end{minipage}
	\hfill	
	% Beliebiger Encoder
	\begin{minipage}{4.7cm}
		\includegraphics[scale=.5]{beliebig.png}
		~\\
		\begin{itemize}
			\item<3-> Liefert meist gute Lösung
			\item<6-> Algorithmisches Vorgehen
			\item<7-> Schnell
			
		\end{itemize}
	\end{minipage}
\end{frame}

\begin{frame}
	\frametitle{Das Projekt}
	
	\begin{itemize}
		\item<+-> Ein Multimedia-Framework zur Evaluierung von Videoencodern
		\item<+-> Videos im YUV-Format einlesen und mit Syntheseoptionen verändern
		\item<+-> Videos mit Hife des Programms analysieren
	\end{itemize}
	\onslide<4>{$ \Longrightarrow $ Programm soll Entwicklern beim Testen ihrer Encoder helfen}
\end{frame}

\begin{frame}
	\frametitle{Die Lösung: YUV.KA}
	\begin{center}
		\includegraphics[height=.9\textheight]{startup_screenshot.png}
	\end{center}
\end{frame}

\begin{frame}
	\frametitle{Planung I}
	Videomanipulation ~\\
	\begin{itemize}
		\item<+-> Eingabe von Videos, Bildern, Farbe und Noise
		\item<+-> Farbkanäle trennen und Videos mergen
		\item<+-> Videos weichzeichnen
		\item<+-> Differenzvideos bilden
		\item<+-> Videoanalyse
	\end{itemize}
\end{frame}

\begin{frame}
	\frametitle{Planung II}
	
	Videoanalyse ~\\
	\begin{itemize}
		\item<+-> Histogramm
		\item<+-> Anzahl der Differenzen der Pixelfarben
		\item<+-> Anzahl der Artefakte und Artefaktoverlay
		\item<+-> Differenz der Encoderentscheidungen
	\end{itemize}
\end{frame}

\begin{frame}
	\frametitle{Planung III}
	
	Sonstige Features ~\\
	\begin{itemize}
		\item<+-> Erweiterbarkeit der Knotenmenge durch Plug-ins
		\item<+-> Parallelisiertes Rendering
	\end{itemize}
\end{frame}

\begin{frame}
	\frametitle{Entwurf I}
	
	Entwicklungsumgebung ~\\
	\begin{itemize}
		\item<+-> C\#
		\item<+-> WPF (\textbf{W}indows \textbf{P}resentation \textbf{F}oundation)
		\item<+-> Microsoft Visual Studio 2010
	\end{itemize}
\end{frame}

\begin{frame}
	\frametitle{Entwurf II}
	\noindent
	\begin{minipage}{3.5cm}
	    \includegraphics[width=3.5cm]{MVVM_thumb.png} ~\\
	    \textit{http://reedcopsey.com}
	\end{minipage}
	\hfill
	\begin{minipage}{8cm}
		Architektur - MVVM-Entwurfsmuster ~\\
	    \begin{itemize}	    
	    	\item<+-> Model-View-ViewModel
	        \item<+-> Eigens für WPF entwickeltes Entwurfsmuster
	        \item<+-> Erzielt Modularität und Testbarkeit durch vollständige Entkopplung von UI-Elementen und UI-Logik
	    \end{itemize}
	\end{minipage}
\end{frame}

\begin{frame}
    \frametitle{Implementierung}
    \noindent
    \begin{minipage}{3.5cm}
        \includegraphics[scale=0.37]{Layers.png}
    \end{minipage}
    \hfill
    \begin{minipage}{8cm}
    \begin{itemize}
        \item<+-> Architektur größtenteils beibehalten
        \item<+-> Entwurf stellte sich als gut machbar heraus
        \item<+-> Änderungen bestehen größtenteils aus Implementierungsdetails ~\\ ~\\
        Darunter fallen:
        \begin{itemize}
            \item<+-> View-spezifische Klassen
            \item<+-> Framework-spezifische Klassen
            \item<+-> Schwer vorhersehbare Properties und Methoden, die sich erst bei der Implementierung als notwendig herausgestellt haben
        \end{itemize}
    \end{itemize}
    \end{minipage}
\end{frame}

\begin{frame}
    \frametitle{Testphase I}
    \begin{itemize}
        \item<+-> Großteil des Projektcodes durch Tests abgedeckt
        \item<+-> Aufbereitung der Daten für die View auf Korrektheit überprüft
        \item<+-> Darstellung der View jedoch nicht abgedeckt, da die View keine Logik enthält
        \item<+-> Korrektheit mancher Klassen nicht automatisch verifizierbar
        \item<+-> Manuelle Überprüfung der Resultate solcher Klassen (bsp. YuvEncoder)
    \end{itemize}
\end{frame}

\begin{frame}
    \frametitle{Testphase II}
    \vspace{1cm}
	\begin{tabular}{@{\extracolsep{\fill}} |l|c|}
		\hline
		Namespace &  Überdeckung in \% \\ \hline
		YuvKA.Pipeline  &  98,32  \\ \hline
		YuvKA.VideoModel  & 86,50 \\ \hline
		YuvKA.ViewModel  & 92,77  \\ \hline
		YuvKA.ViewModel.PropertyEditor  & 98,15  \\ \hline
		YuvKA.Implementation  &  72,15 \\ \hline
		YuvKA.Pipeline.Implementation  &  92,39  \\ \hline
		YuvKA.ViewModel.Implementation  & 84,70 \\ \hline
		YuvKA.ViewModel.PropertyEditor.Implementation  & 76,28  \\ \hline
		\hline
		\textbf{Overall} & \textbf{90,79} \\ \hline
	\end{tabular}
	~\\
	Zeilen eigener Code insgesamt: ca. 7300 ~\\ % actual 7318   
	Zeilen Testcode insgesamt: ca. 3900    % actual 3860
\end{frame}

\begin{frame}
    \frametitle{Erkenntnisse aus dem Projekt}
    \begin{itemize}
        \item<+-> Entwurf vor Implementierung funktioniert...
        \item<+-> ... kann jedoch Probleme mit sich bringen
        \item<+-> Kommunikation und Koordination sind essentiell
        \item<+-> Softwareentwicklung ist sehr zeitintensiv
        \item<+-> (Graphische) Benutzeroberfläche benötigt (wesentlich) mehr Aufwand als Programmlogik
    \end{itemize}
\end{frame}

\begin{frame}
	\begin{center}
		\textbf{Live-Demo}
	\end{center}
\end{frame}

\end{document} 