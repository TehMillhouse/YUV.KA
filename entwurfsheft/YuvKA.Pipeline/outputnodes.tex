\subsection{Ausgabeknoten}

\includegraphics[width=\textwidth]{YuvKA.Pipeline/outputnodes.png}
Die folgenden Klassen stellen Knoten zur Verfügung, die die Ausgabe und Analyse der Videos der Pipeline übernehmen. All diese Klassen erben von der abstrakten Klasse \name{OutputNode}, welche wiederum von der abstrakten Klasse \name{Node} erbt.

\subsubsection{YuvKA.Pipeline.OutputNode}

\begin{verbatim}
public abstract class OutputNode : Node
\end{verbatim}

\paragraph{Beschreibung}~\\
Die abstrakte Klasse \name{OutputNode} modelliert die Ausgabemöglichkeiten der Pipeline und bietet eine gemeinsame Basis für deren konkrete Implementationen.

\subsubsection{YuvKA.Pipeline.DiagramNode}
\begin{verbatim}
[DataContract]
public class DiagramNode : OutputNode
\end{verbatim}
\paragraph{Beschreibung}~\\
Die Klasse \name{DiagramNode} berechnet die Daten, die vom \name{ViewModel} gebraucht werden, um Analysegrafiken zu erstellen. Diese Daten beziehen sich auf die Differenzen der Pixelfarben zweier Frames, das Intra-Block-Verhältniss pro \name{Frame} und den Peak signal-to-noise ratio.

\paragraph{Typmember}
\begin{itemize}

\property{IsEnabled}
	\begin{verbatim}
		[DataMember]
		[DisplayName("Enabled")]
		public bool IsEnabled { get; set; }
	\end{verbatim}
	Setzt fest, ob der Knoten aktiv sein soll, d.h. ob er die \name{Frame}s analysieren soll, oder nicht.
	
\property{ReferenceVideo}
	\begin{verbatim}
		[DataMember]
		[Browsable(false)]
		public Input ReferenceVideo { get; set; }
	\end{verbatim}
	Ruft den Eingang des Referenzvideos, das vom \name{ViewModel} übergeben wurde, ab oder legt ihn fest.

\property{Graphs}
	\begin{verbatim}
		[DataMember]
		[Browsable(false)]
		public List<DiagramGraph> Graphs { get; set; }
	\end{verbatim}
	Stellt eine Liste der ausgewählten/anzuzeigenden Graphen zur Verfügung. (siehe Klasse \name{DiagramGraph})
	
\method{ProcessCore}
	\begin{verbatim}
		public override void ProcessCore(Frame[] inputs, int tick)
	\end{verbatim}
	Berechnet die Daten für die ausgewählten Graphen indem es \name{Process} auf jedem Graph aufruft. (siehe Interface \name{IGraphType})
	
\end{itemize}

\subsubsection{YuvKA.Pipeline.DiagramGraph}
\begin{verbatim}
[DataContract]
public class DiagramGraph
\end{verbatim}

\paragraph{Beschreibung}~\\
Die Klasse \name{DiagramGraph} stellt ein Diagramm dar. Ein Diagramm wird durch seinen Typ, seinen zugehörigen Analysedaten und dem zu analysierenden Video beschrieben.

\paragraph{Typmember}
\begin{itemize}

\property{Video}
	\begin{verbatim}
	[DataMember]
	public YuvKA.Pipeline.Node.Input Video { get; set; }
	\end{verbatim}
	Ruft das zu analysierende Video ab oder legt es fest.
	
\property{Data}
	\begin{verbatim}
	[DataMember]
	public IList<double> Data { get; set; }
	\end{verbatim}
	Enthält eine Liste der berechneten Analysedaten zum entsprechenden Diagramm\name{typ}en.
	
\property{Type}
	\begin{verbatim}
	[DataMember]
	public IGraphType Type { get; set; }
	\end{verbatim}
	Setzt die Art des Analysediagramms fest. (siehe Interface \name{IGraphType})

\end{itemize}

\subsubsection{YuvKA.Pipeline.IGraphType}
\begin{verbatim}
public interface IGraphType
\end{verbatim}

\paragraph{Beschreibung}~\\
Das Interface \name{IGraphType} stellt den Diagrammtypen der Klasse \name{DiagramGraph} dar.

\paragraph{Typmember}
\begin{itemize}

\property{DependsOnReference}
	\begin{verbatim}
	bool DependsOnReference { get; } 
	\end{verbatim}
	Ruft ab, ob der Diagrammtyp das Referenzvideo benötigt.
	
\method{Process}
	\begin{verbatim}
	double Process(Frame frame, Frame reference)
	\end{verbatim}
	Berechnet die Analysedaten des \name{Frame}s und falls vorhanden, berechnet diese im Zusammenhang mit dem \name{Frame} des Referenzvideos. Es werden standardmäßig drei Arten von Bildanalysen unterstützt:
		\begin{itemize}
			\item \name{PixelDiff} Hierbei werden die Farbdifferenzen zwischen dem ausgewählten Frame und dem Referenzframe pixelweise berechnet.
			\item \name{IntraBlockFrequency} Berechnung der Anzahl der Intra-Blöcke im Frame.
			\item \name{PeakSignalNoiseRatio} Stellt das Verhältnis zwischen einem Bild in Originalqualität und dessen komprimierte Version, in der die Informationsmenge durch das Komprimieren in unterschiedlichem Maße abgenommen hat, dar. Dieses Verhältnis wird mit Hilfe folgender Formeln beschrieben. \\
			Der ``mittlere quadratische Fehler'' (oder ``mean square error'') für RGB Pixel wird definiert als 
			\[ \text{MSE} = \frac{1}{3mn} \sum_{i = 1}^{m - 1}{ \sum_{j = 0}^{n - 1}{[I(i, j) - K(i, j)]^2} } \]
			wobei $ I $ und $ K $ zwei Bilder sind, von denen das eine die komprimierte Version des anderen ist. Das ``peak signal-to-noise ratio'' wird dann gegeben durch
			\[ \text{PSNR} = 10 \cdot \log_{10}{\frac{\text{MAX}_I^2}{\text{MSE}}} \]
			wobei $ MAX_I $ der größtmögliche Wert ist, den man in einem Pixel speichern kann. Für RGB Pixel (24-bit) ist das $ 2^{24} - 1 $.
		\end{itemize}
		
\end{itemize}

\subsubsection{YuvKA.Pipeline.HistogramNode}

\begin{verbatim}
[DataContract]
public class HistogramNode : OutputNode
\end{verbatim}
\paragraph{Beschreibung}~\\
Die Klasse \name{HistogramNode} stellt die Daten zur Verfügung, die von dem \name{HistogramViewModel} verwendet werden, um die Histogrammgrafiken darzustellen. Es werden für jeden der drei Farbkanäle sowie für für den Farbwert V im HSV-Raum jeweils ein Histogramm unterstützt.

\paragraph{Typmember}
\begin{itemize}
\property{Type}
	\begin{verbatim}
		[DataMember]
		[Browsable(false)]
		public HistogramType Type { get; }
	\end{verbatim}
	Ruft den Histogrammtyp ab. Es werden vier Arten von Histogrammen unterstützt: jeweils einen für jeden der drei Farbkanäle, sowie einen für den Farbwert im HSV-Raum.
	
\property{Data}
	\begin{verbatim}
		[DataMember]
		[Browsable(false)]
		public double[] Data { get; }
	\end{verbatim}
	Ruft ein Array der Größe 256 ab, in dem die Verteilung des Farbkanals oder HSV-Farbwerts im \name{Frame} dargestellt ist. Die Stelle $ i \ (0 \leq i \leq 255) $ im Array speichert die Häufigkeit des Farbwerts $ i $ im \name{Frame} für den gewählten Farb\name{typ} (R, G, B oder V). Der Wert liegt zwischen 0.0 und 1.0, da eine relative Häufigkeit dargestellt wird.

\method{Process}
	\begin{verbatim}
		public override Frame[] Process(Frame[] inputs, int tick)
	\end{verbatim}
	Die Methode \name{Process} berechnet für den angegebenen Histogramm\name{typ} die Häufigkeit der Farbwerte von 0 bis 255 und speichert diese an der entsprechenden Stelle im Array \name{Data}, wie im folgenden Algorithmus zu sehen ist:
	\floatname{algorithm}{Algorithmus}
	\begin{algorithm}[H]
		\caption{Berechnung der relativen Häufigkeit eines Farbtyps}
		\begin{algorithmic}[1]
			\REQUIRE $ PIXELS, $ \COMMENT { Die Menge aller Pixel im Frame.}
			\STATE \COMMENT{ Sei $ f_{t}(p) $ der Wert der Farbe $ t $ im Pixel $ p \ (t \in \{ R, G, B, V \})$, $ t $ festgesetzt in \name{Type}. \\ $ D_i $ entspricht dem \textbf{Data} Array.}
			\FORALL{$ p \in PIXELS $}
				\STATE $ i \gets f_{t}(p) $
				\STATE $ D_i \gets D_i + 1 $
			\ENDFOR
			\FOR{$ i \gets 0 \text{\textbf{ to }} 255 $}
				\STATE $ D_i \gets \frac{D_i}{|PIXELS|} $
			\ENDFOR
			\ENSURE $ D_i, $ \COMMENT { Das Ausgabearray, in dem die Verteilung gespeichert ist.}
		\end{algorithmic}
	\end{algorithm}

\end{itemize}

\subsubsection{YuvKA.Pipeline.OverlayNode}

\begin{verbatim}
[DataContract]
public class OverlayNode : OutputNode
\end{verbatim}

\paragraph{Beschreibung}~\\
Die Klasse \name{OverlayNode} modelliert das überlagern von Videos mit graphischen Darstellungen von Analysedaten.

\paragraph{Typmember}
\begin{itemize}

\property{Type}
	\begin{verbatim}
	[DataMember]
	public IOverlayType Type { get; set; }
	\end{verbatim}
	Ruft die Art der Überlagerung ab, oder legt sie fest. Standardmäßig werden die Markierung von Artefakten im Vergleich mit einem Referenzvideo, das Anzeigen von Bewegungsvektoren sowie eine Anzeige der Makroblockentscheidungen des Encoders unterstützt.

\method{Process}
	\begin{verbatim}
	public override Frame[] Process(Frame[] inputs, int tick)
	\end{verbatim}
	Überlagert den übergebenen \name{Frame} mit den durch \name{Type} spezifizierten Daten und gibt den Überlagerten \name{Frame} zurück. Bei den standardmäßig unterstützen Möglichkeiten handelt es sich um:
	\begin{description}
		\item[ArtifactsOverlay]~\\
			Für diese Überlagerung werden zwei Eingabeframes benötigt, da hier die Artefakte eines \name{Frame}s im Vergleich mit einem anderen hervorgehoben werden.
		\item[BlocksOverlay]~\\
			Für diese Überlagerung wird eine Eingabe in Form eines \name{AnnotatedFrame}s benötigt, da aus den Logdateien die Makroblockentscheidungen des Encoders entnommen werden um damit die Darstellung dieser Entscheidungen über dem \name{Frame} zu legen.
		\item[MoveVectorsOverlay]~\\
			Für diese Überlagerung wird eine Eingabe in Form eines \name{AnnotatedFrame}s benötigt, da aus den Logdateien die erkannten Bewegungsvektoren des Encoders entnommen werden um diese in den \name{Frame} einzuzeichnen.
	\end{description}
	
\end{itemize}
