\subsection{Eingabeknoten}

\includegraphics[width=\textwidth]{YuvKA.Pipeline/inputnodes.png}
UML-Diagramm zum Klassenbaum, welcher die Knoten darstellt, welche als Pipeline-Eingabequellen dienen.

\subsubsection{YuvKA.Pipeline.InputNode}

\begin{verbatim}
[DataContract]
public abstract class InputNode : Node
\end{verbatim}

\paragraph{Beschreibung}~\\
Die abstrakte Klasse \name{InputNode} modelliert die Eingabequellen der Pipeline, und liefert eine gemeinsame Basis für deren konkrete Implementationen.

\paragraph{Typmember}
\begin{itemize}

\property{Size}
	\begin{verbatim}
	[DataMember]
	public Size Size{ get; set; }
	\end{verbatim}
	Spezifiziert die Größe des vom Knoten gelieferten \name{Frame}s in Form eines \name{Size}-Objektes.

\property{FrameCount}
	\begin{verbatim}
	public virtual int FrameCount { get; }
	\end{verbatim}
	Ruft die Anzahl an vom Knoten zu liefernden \name{Frame}s ab. Der Standardrückgabewert ist 1, die Methode ist allerdings in Unterklassen überschreibbar.

\end{itemize}

\subsubsection{YuvKA.Pipeline.ImageInputNode}

\begin{verbatim}
[DataContract]
public class ImageInputNode : InputNode
\end{verbatim}

\paragraph{Beschreibung}~\\
Die Klasse \name{ImageInputNode} stellt die Funktionalität eines Knoten dar, der lediglich ein aus einer Datei im PNG-Format gelesenes Standbild in die Pipeline einspeist. Sie erbt von der abstrakten \name{InputNode}-Klasse.

\paragraph{Typmember}
\begin{itemize}

\property{FileName}
	\begin{verbatim}
[DisplayName("File Name")]
[DataMember]
public FilePath FileName { get; set; }
	\end{verbatim}
	Entspricht dem Dateipfad der zu benutzenden Datei.

\method{ProcessFrame}
	\begin{verbatim}
	public override Frame[] ProcessFrame(Frame[] inputs, int frameIndex)
	\end{verbatim}
	Ignoriert jegliche Eingabe und liefert als Rückgabewert einen Array, der einen \name{Frame} mit den Bilddaten der Datei (aus dem Pfad \name{FileName}) enthält.

\end{itemize}
\subsubsection{YuvKA.Pipeline.ColorInputNode}

\begin{verbatim}
[DataContract]
public class ColorInputNode : InputNode
\end{verbatim}

\paragraph{Beschreibung}~\\
Die Klasse \name{ColorInputNode} liefert die Funktionalität, einen Videostream einer Benutzerdefinierten Farbe in Form von \name{Frame}s an die Pipeline zu liefern. Sie erbt von der abstrakten \name{InputNode}-Klasse.

\paragraph{Typmember}
\begin{itemize}

\property{Color}
	\begin{verbatim}
	[DataMember]
	public Rgb Color { get; set; }
	\end{verbatim}
	Entspricht der als Quellwert benutzten Farbe.


\method{ProcessFrame}
	\begin{verbatim}
	public override Frame[] ProcessFrame(Frame[] inputs, int frameIndex)
	\end{verbatim}
	Ignoriert jegliche Eingabe und liefert als Rückgabewert einen Array, der einen \name{Frame} enthält welcher vollends mit der vom Benutzer gewählten Farbe ausgefüllt ist.

\end{itemize}

\subsubsection{YuvKA.Pipeline.NoiseInputNode}

\subsubsection{YuvKA.Pipeline.VideoInputNode}
