\includegraphics[width=\textwidth]{YuvKA.VideoModel/videomodel.png}
Das \name{VideoModel} besteht aus denjenigen Klassen, die die unterschiedlichen Framearten bilden, einschließlich ihrer Hilfsklassen. Außerdem gehört die Klasse \name{YuvEncoder} mit der inneren Klasse \name{Video} dazu, die Videos vom YUV-Format ins RGB-Format umwandelt, mit der das Programm intern arbeitet.

\subsubsection{YuvKA.VideoModel.YuvEncoder}

\begin{verbatim}
public static class YuvEncoder
\end{verbatim}

\paragraph{Beschreibung}~\\
Die Klasse \name{YuvEncoder} bietet die Funktionalität, Videos, die im YUV-Format eingegeben werden, ins RGB-Format umzuwandeln, damit das Programm damit arbeiten kann. Des Weiteren wandelt sie die manipulierten Videos vom RGB-Format ins YUV-Format um, damit sie abgespeichert werden können.

\paragraph{Typmember}
\begin{itemize}

\method{Decode}
	\begin{verbatim}
	public static Video Decode(string fileName, string logFileName,
    int width, int height)
	\end{verbatim}
	Wandelt das im \name{fileName} angegebene YUV-Video ins RGB-Format um, das interne Format des Programms. Falls der Parameter \name{logFileName} nicht \name{null} ist, werden die \name{Frame}s mit entsprechendem Log als \name{AnnotatedFrame}s ausgegeben.

\method{Encode}
	\begin{verbatim}
	public static void Encode(string fileName,
    IEnumerable<Frame> frames)
	\end{verbatim}
	Wandelt das als RGB-Frameliste angegeben Video ins YUV-Format um und speichert es unter \name{fileName}. 

\end{itemize}

\setcounter{secnumdepth}{4}
\paragraph{YuvKA.VideoModel.YuvEncoder.Video}
\setcounter{secnumdepth}{3}

\begin{verbatim}
public class Video : IDisposable
\end{verbatim}

\paragraph{Beschreibung}~\\
Die Klasse \name{Video} hält die vom \name{YuvEncoder} umgewandelten Videos als Videostream.

\paragraph{Typmember}
\begin{itemize}

\property{FrameCount}
	\begin{verbatim}
	public int FrameCount { get; }
	\end{verbatim}
	Ruft die Anzahl der \name{Frame}s im \name{stream} ab.

\property{this[int index]}
	\begin{verbatim}
	public Frame this[int index] { get; }
	\end{verbatim}
	Ruft den an der Stelle \name{index} abgespeicherten \name{Frame} im \name{stream} ab.

\method{Dispose}
	\begin{verbatim}
	public void Dispose()
	\end{verbatim}
	Schließt den \name{stream} des \name{Video}-Objekts, sodass das File-Handle freigegeben wird.

\end{itemize}

\subsubsection{YuvKA.VideoModel.Frame}

\begin{verbatim}
public class Frame
\end{verbatim}

\paragraph{Beschreibung}~\\
Die Klasse \name{Frame} stellt ein Einzelbild eines Videostreams im RGB-Format dar.

\paragraph{Typmember}
\begin{itemize}

\property{Size}
	\begin{verbatim}
	public Size Size { get; }
	\end{verbatim}
	Ruft die Auflösung des \name{Frame}s ab.

\property{this[int x, int y]}
	\begin{verbatim}
	public Rgb this[int x, int y] { get; set; }
	\end{verbatim}
	Ruft den Farbwert des \name{Frame}s an der Stelle \name{x,y} in RGB ab oder legt ihn fest.

\end{itemize}

\subsubsection{YuvKA.VideoModel.AnnotatedFrame}

\begin{verbatim}
public class AnnotatedFrame : Frame
\end{verbatim}

\paragraph{Beschreibung}~\\
Die Klasse \name{AnnotedFrame} erweitert die Klasse \name{Frame} um die Funktion, Logfiles des Encoders zu halten.

\paragraph{Typmember}
\begin{itemize}

\property{Decisions}
	\begin{verbatim}
	public MacroblockDecision[] Decisions { get; }
	\end{verbatim}
	Ruft die Entscheidungen, die der Encoder bei der Kodierung getroffen hat, als Array von \name{MacroblockDecision}-Objekten ab.

\end{itemize}

\subsubsection{YuvKA.VideoModel.MacroblockDecision}

\begin{verbatim}
public class MacroblockDecision
\end{verbatim}

\paragraph{Beschreibung}~\\
Die Klasse \name{MacroblockDecision} stellt die Entscheidung dar, die der Encoder bei der Enkodierung eines Makroblocks getroffen hat.

\paragraph{Typmember}
\begin{itemize}

\property{Movement}
	\begin{verbatim}
	public Vector Movement { get; }
	\end{verbatim}
	Ruft den Bewegungsvektor des Makroblocks relativ zum vorherigen Frame ab.

\property{PartitioningDecision}
	\begin{verbatim}
	public MacroblockPartitioning PartitioningDecision { get; }
	\end{verbatim}
	Ruft die Entscheidung darüber, welche \name{MacroblockPartitioning} bei der Kodierung verwendet wurde ab.

\end{itemize}

\subsubsection{YuvKA.VideoModel.MacroblockPartitioning}

\begin{verbatim}
public enum MacroblockPartitioning
{
    InterSkip, Inter16x16, Inter16x8, Inter8x16,
    Inter8x8, Inter8x4, Inter4x8, Inter4x4,
    Inter8x8OrBelow, Intra4x4, Intra16x16, Intra8x8,
    IntraPCM, Unknown
}
\end{verbatim}
Die \name{MacroblockPartitioning} gibt die verschiedenen Möglichkeiten der Partitionierung eines Makroblocks als Inter- bzw. Intraframe an.
