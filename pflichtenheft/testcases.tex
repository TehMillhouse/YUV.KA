\section{Testfälle}

\textbf{Folgende Funktionssequenzen sind zu überprüfen:}
\begin{itemize}
	\item\textbf{/T10/} Pipeline-Konstruktion
		\begin{itemize}
			\item Der Benutzer startet das Programm.
			\item Der Benutzer wählt "Neue Pipline erstellen".
			\item Der Benutzer erzeugt jede Art von Knoten mindestens einmal, indem er jeden Knoten per "Drag-and-Drop"\ aus einer am unteren Bildschirmrand
				angebrachten Leiste erzeugt.
			\item Der Benutzer verbindet die erstellten Knoten miteinader, indem er Kanten zwischen den Aus- und Eingängen der Knoten erstellt.
		\end{itemize}
	\item\textbf{/T20/} Manipulation einer Pipeline
		\begin{itemize}
			\item Der Benutzer startet das Programm und erstellet eine beliebige Pipeline mit mindestens 2 Knoten, davon mindestens ein Manipulationsknoten mit Optionen 
				und mindestens einer Kante.
			\item Der Benutzer bewegt mindestens einen gesetzte Knoten innerhalb der Pipeline per "Drag-and-Drop".
			\item Der Benutzer verändert das Ziel mindestens einer gesetzten Kante per "Drag-and-Drop".
			\item Der Benutzer löscht mindestens eine Kante.
			\item Der Benutzer verändert mindestens eine Einstellung eines Manipulationsknotens.
			\item Der Benutzer löscht mindestens einen Knoten.
		\end{itemize}
	\item\textbf{/T30/} Sicherung einer Pipeline
		\begin{itemize}
			\item Der Benutzer startet das Programm und erstellet eine beliebige Pipeline.
			\item Der Benutzer speichert die Pipeline.
			\item Der Benutzer wählt "Neue Pipeline erstellen".
			\item Der Benutzer lädt die gesicherte Pipeline.
		\end{itemize}
\newpage
	\item\textbf{/T40/} Videoverarbeitung
		\begin{itemize}
			\item Der Benutzer startet das Programm und erstellet eine beliebige, zusammenhängende, zyklenfreie Pipeline mit mindestens einem Eingabe-, Wiedergabe- sowie 
				Manipulationsknoten mit Optionen.
			\item Der Benutzer ändert die Quelle eines Eingabeknotens.
			\item Der Benutzer öffnet den Wiedergabeknoten und beginnt das Video abzuspielen.
			\item Der Benutzer ändert mindestens eine Option eines Manipulationsknotens, während das Video abspielt.
			\item Der Benutzer ändert die Abspielgeschwindigkeit des Videos.
			\item Der Benutzer pausiert die Videowiedergabe.
			\item Der Benutzer setzt die Wiedergabe fort.
			\item Der Benutzer setzt die Videowiedergabe zurück.
			\item Der Benutzer speichert das manipulierte Video als YUV-Datei.
		\end{itemize}
	\item\textbf{/T50/} Videoanalyse
		\begin{itemize}
			\item Der Benutzer startet das Programm und erstellet eine beliebige, zusammenhängende, zyklenfreie Pipeline mit mindestens einem Eingabe-, Überlagerungs- sowie 
				Diagrammknoten.
			\item Der Benutzer deaktiviert den Diagrammknoten.
			\item Der Benutzer öffnet den Überlagerungsknoten.
			\item Der Benutzer beginnt die Videowiedergabe.
			\item Der Benutzer fügt eine Überlagerungsoption hinzu, während das Video abspielt.
			\item Der Benutzer entfernt die hinzugefügte Überlagerungsoption.
			\item Der Benutzer setzt die Videowiedergabe zurück und schließt den Überlagerungsknoten.
			\item Der Benutzer reaktiviert den Dieagrammknoten und öffnet diesen.
			\item Der Benutzer wählt ein Referenzvideo aus und fügt einen Analysegraphen hinzu.
			\item Der Benutzer startet erneut die Videowiedergabe.
			\item Der Benutzer ändert den Typ des Analysegraphen.
			\item Der Benutzer löscht den Analysegraphen.
		\end{itemize}
\end{itemize}

\newpage

\textbf{Folgende Datenkonsistenzen sind einzuhalten:}
\begin{itemize}
	\item\textbf{/T100/} Ergebnislose Wiedergabe verhindern. ~\\
		Wenn der Benutzer keinen Endknoten geöffnet hat, ist die Videowiedergabe nicht anwählbar.
	\item\textbf{/T110/} Verarbeitung ohne Eingabe verhindern. ~\\
		Falls ein Eingabeknoten ohne gültige Videoquelle mit der Pipeline verbunden ist, ist weder die Videowiedergabe, noch das Speichern von Videos als YUF-Datei möglich.
	\item\textbf{/T120/} Strukturbrüche während der Wiedergabe verhindern ~\\
		Während der Videowiedergabe kann der Benutzer die Pipelinestruktur nicht verändern.
	\item\textbf{/T130/} Inkonsistenz eines gespeicherten Videos verhindern ~\\
		Während dem Speichern eines Videos als YUF-Datei kann der Benutzer keinerlei Änderungen an der Pipeline vornehmen.
\end{itemize}
