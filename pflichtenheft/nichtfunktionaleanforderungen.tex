\section{Nichtfunktionale Anforderungen}

\subparagraph{Benutzbarkeit}
\begin{description}
	\item[/NF10/] Die Benutzeroberfläche soll einfach und intuitiv zu bedienen sein. Unerfahrene Benutzer sollen ohne Vorwissen in der Lage sein, sich schnell in die Funktionalitäten einzuarbeiten.
	\item[/NF20/] Die Zusammenstellung sowie die Nacheinanderreihung der verschiedenen Manipulationseffekte soll einfach und übersichtlich geschehen können. 
	\item[/NF30/] Der Benutzer muss in der Lage sein, neue Manipulationseffekte hinzuzufügen, ohne die Reihenfolge der bereits bestehenden ändern zu müssen.
	\item[/NF40/] Im Programm sollen theoretisch gleichzeitig uneingeschränkt viele Videos verwaltet, bearbeitet und analysiert werden können.
	\item[/NF50/] Der Benutzer muss ein Video aus jeder Stufe der Bearbeitung problemlos abspielen, analysieren oder exportieren können.
\end{description}

\subparagraph{Geschwindigkeit}

\begin{description}
	\item[/NF100/] Bei einer einfachen analysefähigen Pipeline mit einem einzigen Manipulationseffekt sollen ein Video der Auflösung 320p und die zugehörigen Statistiken in angemessener Geschwindigkeit
 ausgegeben werden können.
\end{description}

\subparagraph{Robustheit}

\begin{description}
	\item[/NF200/] Das Programm soll trotz falscher Benutzereingaben oder Parameter nicht abstürzen. Es muss in diesen Fällen entsprechend reagieren.
\end{description}

\subparagraph{Erweiterbarkeit}

\begin{description}
	\item[/NF300/] Das Programm soll sich einfach um neue Funktionalitäten erweitern lassen.
\end{description}
