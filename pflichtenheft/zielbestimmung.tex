\section{Zielbestimmung}

Der Benutzer soll durch das Produkt in die Lage versetzt werden, Videoencoder zu testen, indem Videos manipuliert, encodiert und anhand eines Referenzvideos verglichen werden.

\subsection{Musskriterien}

\begin{itemize}
	\item Eingabe von unkomprimierten Videos, einer festen Farbe, einem Bild oder Noise.
	\item Manipulation der Videos
	\begin{itemize}
		\item Erstellen einer Pipeline mit verschiedenen Manipulationsoptionen.
		\item Zusammenstellung über knotenbasierte GraphView
	\end{itemize}
	\item Analyse der Videos
	\begin{itemize}
		\item Anzeigen verschiedener Diagramme.
		\item Videoausgabe mit optionalen Overlayoptionen.
	\end{itemize}
	\item Speichern von Videos und Pipeline.
\end{itemize}

\subsection{Wunschkriterien}

\begin{itemize}
	\item Benutzern eine Undo/Redo-Funktion bieten.
	\item Verwenden von Tabs im Interface.
	\item Benutzern die Möglichkeit bieten, die Aufösung der eingegebenen Videos automatisch zu erkennen.
	\item Verwenden einer asynchronen, nichtblockenden UI.
	\item Ausnutzen von Multithreading-Möglichkeiten.
	\item Verarbeiten der Motionvektoren des Encoders.
\end{itemize}

\subsection{Abgrenzungskriterien}

\begin{itemize}
	\item Keine Soundausgabe
	\item Nur YUV-Videos als Eingabe zugelassen
	\item Keine Speicherung der Analysedaten
\end{itemize}
