\renewcommand{\labelitemi}{$-$}
\section{Zielbestimmung}

Der Benutzer soll durch das Produkt in die Lage versetzt werden, Videoencoder zu testen, indem Videos manipuliert, encodiert und anhand eines Referenzvideos verglichen werden.

\subsection{Musskriterien}

\begin{description}
	\item[Eingabe] Der Benutzer soll unkomprimierte Videos, eine feste Farbe, ein Bild oder Noise zur Eingabe verwenden können.
	\item[Manipulation der Videos] Der Benutzer soll eine Pipeline mit verschiedenen Manipulationsoptionen erstellen können, die über eine knotenbasierte Benutzeroberfläche zusammengestellt wird.
	\item[Analyse der Videos] Der Benutzer soll verschiedene Diagramme anzeigen können und Videos mit optionalen Overlayoptionen abspielen können.
	\item[Ausgabe] Der Benutzer soll die Videos und Pipeline speichern können.
\end{description}

\subsection{Wunschkriterien}

\begin{itemize}
	\item Anbieten einer Undo/Redo-Funktion
	\item Anbieten von Tabs im Interface
	\item Anbieten einer automatischen Auflösungserkennung für eingegebene Videos
	\item Verwenden einer asynchronen, nichtblockenden UI
	\item Ausnutzen von Multithreading-Möglichkeiten
	\item Verarbeiten der Motionvektoren des Encoders
	\item Erweiterbarkeit der Knotenmenge durch Plugins
\end{itemize}

\subsection{Abgrenzungskriterien}

\begin{itemize}
	\item Keine Soundausgabe
	\item Nur YUV-Videos als Eingabe zugelassen
    \item Keine Unterstützung für Bildtransparenz
	\item Keine Speicherung der Analysedaten
\end{itemize}
